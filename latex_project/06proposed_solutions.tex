\chapter{Proposed solutions}\label{chap:proposed_solutions}

	Motivated by previous solutions, mentioned in Chapter \ref{chap:existing_solutions}, Sections \ref{subsec:intra_inter} and \ref{subsec:linear_geo}, we propose multiple approaches to solve the problem of tracklet building. The first one, described in Section \ref{sec:linear_regression} takes advantage of the facts mentioned in Section \ref{subsec:linear_geo}, specifically that picking sufficiently small part of the object's trajectory will make the object appear as if it moved in uniform linear motion. Section \ref{sec:IDO} describes an algorithm which filters and validates data gathered in Section \ref{sec:linear_regression}. An experimental neural network's construction and relevance is discussed in Section \ref{sec:neural} and Section \ref{sec:hough} acknowledges another possible solution which has not been implemented but has been considered nevertheless.

\section{Use of linear regression}\label{sec:linear_regression}

	Linear regression is a well-known statistical concept modelling the linear relationship between a scalar dependent variable \emph{y} and one or more independent variables, usually denoted as \emph{x}. In this thesis, we will be using a single scalar predictor variable and a single scalar response variable - simple linear regression (SLR). We chose SLR because objects in our case appear as if they followed the rules of uniform linear motion (for details on uniform linear motion, see Chapter \ref{chap:existing_solutions}, Section \ref{sec:linear_motion}). For more information about linear regression as a statistical model, see \citep{freedman2005statistical}.
	
	To successfully create a tracklet, we need to have three or more confirmed observations out of several images. As described in Chapter \ref{chap:object_dynamics}, Section \ref{sec:proc_seg_reduc} all objects are classified beforehand and have several mandatory attributes assigned to them.
	
	As discussed in Chapter \ref{chap:object_dynamics}, Section \ref{sec:ccd} objects' position on the image reference frame in each image in the set of provided inputs (see Chapter \ref{chap:requirements}, Section \ref{sec:input_data}) are represented by the values \emph{x} and \emph{y}. The first draft of linear regression was designed and realised using these values - \emph{x} as an independent value and \emph{y} as a dependent value.
	
	There are several unidentified or misidentified objects which are potentially the object we are looking for, represented as points. 
	
	Firstly, we pair each unidentified point $p_1$ of an object $o_1$ from the first image with each unidentified point $p_2$ of an object $o_2$ from the second image. Then we create a line $l$ such that $p_1,\ p_2 \in l$. The final number of existing lines after this initial procedure is equal to $n_1 * n_2$ where $n_1$ is the number of unidentified points in the first image and $n_2$ is the number of unidentified points in the second image. It is important to note that this number might be relatively high depending on the number of unidentified points and therefore on the quality of the pre-processing of each image. If we ignored all fake points we would be left with one line placed among all the real points from the rest of the images.
	
	However, there are several problems. First, fake objects might appear along a line in the same fashion as the real do. Second, the real points might have non-negligible distance from a line. Third, the first or second point might be deviated too - this would cause the line to have incorrect slope. Fourth, the real points might appear as if their acceleration was non-zero. 
	
	The last three problems are mainly caused by atmospheric fluctuations and errors in the CCD camera processing. We provide solutions to all of these problems in the next paragraph.
	
	\begin{figure}[H]
	\centering
	  \includegraphics[width=12cm]{images/regresia1}
		  \caption{Points scattered along a trend line.}
	  \label{fig:regresia1}
	\end{figure}
	
	Let's imagine we have a single line with points scattered around it, as is illustrated in Figure \ref{fig:regresia1}. For clarity, the figure shows all points gathered from all images. To further filter out points which have almost zero probability of being real objects we introduce a threshold $T_l$ which determines a maximum distance a point can have from the line. 
	
	The distance of a point $p=(x,y)$ from the line $Am+Bn+C=0$ is calculated by using formula $$d_p=\frac{|Ax+By+C|}{\sqrt{A^2+B^2}}$$ We disregard a point if $d_p>T_l$, otherwise we process it in the next filter stage. Using only this criterion we can still be left with many false positives and thus, we extend our filter by calculating angle between the currently considered point and the last most probable point. 
	
	Consider having two points represented by its positions $p_{i}=(x_i,y_i)$, $p_j=(x_j,y_j)$. We calculate angle $\theta$ between them as $$\theta=|\arctantwo(y_j-y_i, x_j-x_i)|$$ We then check whether the angle is in a predetermined threshold $T_a$. By doing this we create an imaginary 2D cone forming from each point to the plane, as is illustrated in Figure \ref{fig:regresia2}.
	
	\begin{figure}[H]
	\centering
	  \includegraphics[width=12cm]{images/regresia2}
		  \caption{Heading of an object.}
	  \label{fig:regresia2}
	\end{figure}
	
	%pravdepodobne zmena algoritmu, kedy sa bude pozerat currently considered point and second point uhol
	
	The last stage of the filter contains determination of speed $s$, which is calculated by using the formula $$s=\frac{d}{{\Delta}t}$$ where $$d=\sqrt{(x_1-x_2)^2 + (y_1-y_2)^2}$$ where $(x_1,y_1)$ and $(x_2,y_2)$ are two successive points in time and $${\Delta}t=|t_1-t_2|$$ We get the time $t_{i},\ i\in\langle1,N\rangle$ where $N$ is the number of images. 
	
	Only after each of these three filters has been satisfied by the currently chosen object $o_i$ and its point $p_i$, we add it to the list of potential candidates. 
	
	However, there is yet another metric that helps determining the most probable object for a currently constructed tracklet. Filters only select out points within user set boundaries - the candidates are then sorted by comparing the values gotten while processing filters. We treat first two objects $o_1$ and $o_2$ as the baseline - we calculate their speed and angle and then compare speed and angle of each successive object $o_i$. The object with the closest value is then put at the front of the list and therefore marked as the most probable real object. SLR is then calculated again, containing the last added object as well. It is important to note that we add exactly one object from every image before moving to the next image. As opposed to adding every filtered object this has proved to be a better solution.
	
	We touched upon every problem mentioned above - we disregarded points which are too far from the line, we correct the line by dynamically adding objects and re-calculating SLR, we filter out fake objects with nonsense speeds and angles. However, the situation when a fake object passes through all these filters and is marked as real is possible and is remedied, or rather verified, by using IOD.
	
	The representation of each object $o_i$ can be specified either by the standard coordinate system in the two-dimensional Euclidean plane or by a reference system, specifically RA/Dec (see Chapter \ref{chap:object_dynamics}, Section \ref{sec:ra_dec}). The main difference and the reason why we swapped a coordinate system for a reference system is that the RA/Dec is precise and mostly devoid of errors and deviations. The difference is best shown rather than explained - see Figure \ref{fig:regresia3}. On the left, there are shown 8 observations of the same object (an asteroid 2017\_PR25), represented in modified RA/Dec and on the right the same 8 observations, represented in the standard coordinate system. It is clearly visible that the deviation of each point in the right graph is more extreme than in the left graph. Acceleration of some of the points appears to be higher than zero - this is because the images were taken in different time intervals.
	
	\begin{figure}[H]
	\centering
	  \includegraphics[width=\linewidth]{images/regresia3}
		  \caption{RA/Dec comparison.}
	  \label{fig:regresia3}
	\end{figure}

\section{Use of the Initial Orbit Determination algorithms}\label{sec:IDO}

	The theoretical details of IOD are described in Chapter \ref{chap:object_dynamics}, Section \ref{sec:init_orbit_det}. We used a fully functional implementation of IOD from \citep{Silha2012id}. For functioning correctly, the algorithm requires longitude and latitude in radians, altitude (in our case AGO - see Chapter \ref{chap:introduction}, Section \ref{subsec:fmpi_ago}), RA/Dec in radians and time of the observation in Modified Julian Date (MJD) of three different objects. Then, we perform both Montenbruck and Escobal methods of IOD.

\section{Use of neural network}\label{sec:neural}

	An experimental part of this thesis is the use of a neural network. We are using a popular framework \emph{tflearn}.

\section{Hough transform}\label{sec:hough}
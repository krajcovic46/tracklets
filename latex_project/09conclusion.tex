\chapter{Conclusion}\label{chap:conclusion}

	In this thesis we have presented the need for the construction of tracklets, base knowledge concerning space debris objects and their physical features and their use in today's astronomical scientific community. We have analysed currently professionally used algorithms for tracklet building employed in many world's respected establishments. 	
	
	We have successfully designed and built our own relatively simple, yet powerful and elegant system for filtering out fake objects, creating tracklets and displaying results in a coherent, both visual and textual, way.
	
	The system will be implemented on the server in a pipeline at the Astronomical and geophysical observatory in Modra where it will be further tested, improved and used to construct tracklets from the images of the night sky.
	
	The room for improvement lies mainly in the machine learning part. While we were unsuccessful in training the neural network and directing it in such way that it would produce reasonable results, we have learned invaluable lessons which will be employed in next phases of the life-cycle of this system. Other improvements can be made in calculating thresholds and thus refining the SLR algorithm even more and producing cleaner results.